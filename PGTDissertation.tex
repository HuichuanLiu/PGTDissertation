\documentclass[12pt,a4paper]{muthesis}
\usepackage{url}
\usepackage{graphicx}
\usepackage{caption}
\usepackage{algorithm}
\usepackage{algorithmic} 
\usepackage{CJK}
\usepackage{amsmath}


\bibliographystyle{unsrt}

\newcommand{\PreserveBackslash}[1]{\let\temp=\\#1\let\\=\temp}

\newcommand{\HRule}{\rule{\linewidth}{0.5mm}} 
\renewcommand\thesection{\arabic {section}}

\setcounter{secnumdepth}{3}

\begin{document}

\begin{titlepage} 
\begin{center} 
\textsc{\LARGE University of Manchester}\\[1.5cm] 
\textsc{\Large Progress Report}\\[0.5cm]

\HRule \\[0.8cm] { \huge \bfseries \linespread{10}The Application of Semi-supervised Machine Learning Algorithms in Parkinson's Disease Analysis}\\[0.4cm] 

\HRule \\[1.5cm] 
\begin{minipage}{0.4\textwidth} 
\begin{flushleft} 
\large \emph{Author:}\\ Huichuan \textsc{Liu} 
\end{flushleft} 
\end{minipage} 
\begin{minipage}{0.4\textwidth} 
\begin{flushright} 
\large \emph{Supervisor:} \\ Dr. Simon \textsc{Harper} 
\end{flushright} 
\end{minipage} \vfill 
{\large \today} 
\end{center} 
\end{titlepage}

\newpage
\tableofcontents
\newpage
\listoffigures
\newpage
\listoftables


\abstract

This dissertation represents the feasibility of deploying an automatic tracker of Parkinson's Disease (PD) on smartphones. According to the report of The Royal Free Hospital NHS Trust (2008), Parkinson's Disease has become one of the most prevalent neurodegenerative disorder in the Western Europe. However, most diagnoses of PD progress remain in a conventional way, i.e. regular visiting clinics and diagnosed by physicians, which is rather expensive and inconvenient for patients with motor impairment. Hence, the combination of special equipment and machine learning technologies have been developed to accomplish automatic measuring of PD symptoms. However, the applications are inevitably limited by the cost of extra equipment and the requirement of training labels. A potential solution is to introduce smartphones to replace professional devices. Nevertheless, there are still barriers and challenges in practice scenes. As a pilot research in this field, a series of experiments are set in this project to explore three specific problems concerning PD tremor assessment, namely the strategy of data preprocess, the performance of semi-supervised learning algorithms and the assessments of multiple data features. 
 
In the first part, we propose a strategy of data reduction and some signal processes smartphone data, aiming at reducing the computational cost and improving the data quality. Then three semi-supervised learning algorithms are implemented, evaluated and compared regarding the parameter effectiveness and overall performance. Last but not least, features used in different models are assessed by Principle Component Analysis (PCA) respectively in the final stage to reveal the relevance of features and models. 


%---------------------------------------------------------------- Chapter 1 Introduction ----------------------------------------------------------------%

\chapter{Introduction}
Parkinson's Disease (PD), also known as idiopathic or primary Parkinsonism, is a progressive neurological disorder caused by cell loss in the substantia nigra. According to the survey by National Collaborating Centre for Chronic Conditions (Great Britain) in 2006, the prevalence of PD was around 100-180 per 100,000 of the population in the UK\cite{national2006parkinson}. Parkinson's Disease have been recognized as an illness that significantly impairs patients' motor ability and arouses sorts of non-motor discomforts\cite{davie2008review}. Many previous studies\cite{national2006parkinson,davie2008review,izworski2005acquisition,patel2009monitoring,goetz2008movement} indicate that unnatural tremors is an obvious symptom of Parkinson's Disease and the situation of these tremors are closely relative to the progress of the disease. Hence, having effective and accurate measurement of tremors are important for medical research of PD and its clinical treatments of Parkinson's Disease.

%Conventional defects
In fact, a lot of methods have been proposed for PD progress measuring and scaling. The most famous ones include Hoen and Yahr\cite{hoehn1998parkinsonism}, the Unique Parkinson's Disease Rating Scale (UPDRS)\cite{movement2003unified} and Northwestern University Disability Scale (NUDS)\cite{goetz2008movement}. These methods provide questionnaires of PD symptoms and a standard for marking the answers into scores. In the last decades, these conventional ways are proved to be effective and helpful for PD diagnoses. But they are still flawed in terms of objectivity and integrity. Also, these methods literally count on frequently visiting of clinics, hence it is rather inconvenient for patients and quite expensive in human resources.

%Advanced techniques
In order to achieve more effective diagnoses, more advanced instrument and techniques were involved in this area and several exciting achievements have been reported\cite{patel2009monitoring,pan2012parkinson,gil2009diagnosing}. Motor sensors are widely used to record the motor situation of PD patients and data technology, especially machine learning techniques, were introduced to implement automatic diagnose systems. In laboratory environment, a number of experiments have been tried to deliver automatic assessment of PD progress. But in practice scenes, there are still problems to be solve.

First of all, data collection is a difficult part for tremor monitoring because extra devices are costly and uncomfortable for some patients\cite{giuffrida2009clinically}. Secondly, the labeling work to produce training data is time-consuming and expensive. What is more, most existing systems provide only PD tremor detecting but few of them are capable of rating the progress.

%our targets
To overcome these barriers, we proposed a semi-supervised learning system to assess the PD progress from smartphone data. More specifically, we try to apply preprocess to improve the data quality of smartphone sensors, which enables an more natural and economy data collection. Then a semi-supervised learning (SSL) system is implemented to resolve the shortage of labels and predicts the scores/progress of the Parkinson's Disease. For better comprehensiveness, several SSL models will be compared, including regression methods and classification methods.

\section{Objectives and Goals}
The overall target of this project is to explore the application of semi-supervise learning in distinguishing PD tremors with diverse severities from data collected by smartphones. Certain challenges have been allocated and a train of techniques are attempted as resolutions. First, sensor signals from smartphones are of low quality but massive and the information they contain are too obscure. Hence, a preprocessing module is built to accomplish sensor signal de-nosing, data reduction and feature extraction. Next, a control experiment is set to conduct semi-supervise machine learning with processed data. Meanwhile, each model will be evaluated in terms of entropy, accuracy and ROC factors.This stage focuses on the performance of models and the impacts of parameter regulation. Finally, by computing the information gain of each feature in different models, a strategy of feature selection in different models is described.
\begin{itemize}
\item Data Preprocess.
\item[ - ] Data Cleaning: Deploying Data Smoothers to reduce impacts of interferences.
\item[ - ] Data Reduction: Applying rule-based filtering to capture interesting periods from massive raw data.
\item[ - ]  Feature Extraction: Extracting signal features from acceleration data.
\item Semi-supervised Machine Learning. 
\item[ - ]  Implementation: Implementing an semi-logic regression model, an semi-KNN model and a GMM model.
\item[ - ]  Parameter Optimization: Varying the model parameters to reveal the effects of parameters. 
\item[ - ] Evaluation: Evaluating models by system entropy, accuracy and Receiver Operating Characteristic (ROC) factors.
\item Feature Assessment
\item[ - ]  Information Gain Analysis: Training models with diverse feature combination and calculating the information gain of each feature respectively.
\end{itemize}


\section{Dissertation Structure}
This dissertation is composed of 7 chapters. Chapter 1 gives an overview of the motivation and challenges in automatic PD tremor analysis. In Chapter 2, we present some crucial findings of previous research that closely related to our project. Next, the fundamental knowledges and experiment designs are stated in Chapter 4. Chapter 5 illustrates the system implementation and experiment details. Later in Chapter 6, we demonstrate the experiment results and critical interpretations. Then Chapter 7 provides the overall summary and conclusions. In the final chapter, some limitations are pointed out and relevant future work is proposed. 

\section{Summary}
In the first chapter, the challenges and motivation of an SSL-based tremor assessment system is illustrated. Based on that, more detailed objectives, have been listed out. Last but not least, a framework of the whole dissertation is established.

%---------------------------------------------------------------- Chapter 2 Background ----------------------------------------------------------------%

\chapter{Background}
PD tremor Analysis is a comprehensive application crossing multiple domains, including medicine, signal process and data learning. Obviously, early research of PD symptoms provided theoretical basis about descriptors in measurement\cite{goetz2008movement,jankovic1999re,lang1998parkinson,brennan2002essential}. Based on that, certain methods were proposed and experimentally tested, a part of those delivered crucial inspiration and vital information in this field. However, inconvenient defects are also allocated in existing systems. To overcome these defects, we involve semi-supervised learning techniques in our project.

\section{PD and PD tremors}
In general, patients with PD regularly sustain motor impairments and non-motor discomforts. Non-motor symptoms like cognitive impairment, anxious mood, and fatigue dramatically decrease the life quality of patients, but there are few frames could be used to quantify these symptoms in an analyzable way yet. In contrast, motor symptoms would be more observable and measurable by sensors, hence showing great potentials as indicators of PD progress. With the development of the disease, the motor symptoms of PD includes tremors, bradykinesia, rigidity and postural instability\cite{davie2008review}. These motor symptoms are commonly used to scale the progress of PD in clinical diagnoses. Among all motor symptoms, PD tremors are recognized as a typical and apparent descriptor of PD progress. In comparison with other motor symptoms, PD tremors are rather intuitive in observation and quantification because it is relatively independent of other body actions. The vibration changes could be simply tracked and measured by acceleration. On the contrary, many complicate factors would be involved when analyzing bradykinesia, rigidity and postural instability, such as activity recognition, posture identification, and environment analysis. Therefore, recognition and measurement of tremors could be significant and effective in clinical diagnoses of PD\cite{lesage2009parkinson}. 

However, there are multiple types of tremors appearing on human bodies. Rest tremors are regarded as a cardinal symptom whiles re-emerge tremors are also associated with PD\cite{jankovic1999re}. According to Jankovic, re-emerge tremors, which are proved relevant to PD, are only a part of posture tremors. A frequency-based analysis has been described in Jankovic's experiment that capable of distinguishing the re-emerge tremors from other posture tremors \cite{jankovic1999re}. As for kinetic tremors,  some research successfully discriminated them with the posture tremors, and they were thought to be less prevalent than the posture tremors \cite{lang1998parkinson,brennan2002essential}. As for the difference between posture tremors and rest tremors, a previous experiment held by Koller indicated that posture tremors (92\%) are more prevalent than the posture tremors (76\%), and the amplitude of posture tremors are 50\%higher, 25\%the same and 25\% less than the rest tremors. As to frequency, there is no significant difference between rest tremors and posture tremors in that experiment\cite{koller1989tremors}. These explorations implied the complicity of analyzing PD tremors in naive observations and implied the necessity of more skillful techniques in this domain. 

In recent years, signal analysis techniques and machine learning methods are attempted in PD analysis and several characteristics of PD tremors have been presented. The experiment of Robert et al.\cite{lemoyne2010implementation} showed significant diversity in acceleration of hands between PD patients and healthy people. In details, the mean time-averaged acceleration of PD patients was 2.4 times greater than the normal while the coefficient of variation for the PD patients was 4.1 times larger than the normal. Besides, the importances of the range of dominant frequencies and spectral analysis was mentioned.In another project held by Bonato et al., their researchers have successfully allocated five types of acceleration attributes as informative descriptors in PD tremor assessment, namely the range of amplitude of each channel, the root mean square value (RMS) of each frequency, two cross-correlation-based features, two frequency-based features and the signal entropy\cite{bonato2004data,moddemeijer1989estimation}. In details, the cross-correlation-based features include the peak of the normalized cross-correlation function and the time lag of the peaks, while the frequency-based features consist of the dominant frequency component and the corresponding ratio of the energy.  However, according to Patel et al.\cite{patel2009monitoring}, utilizing only three features (RMS, amplitudes, and signal entropy features) resulted in a bottom error rate (2.5\%) in terms of tremor recognition and instead of decrease, adding more features slightly raised the error rate (2.7\% for 4 features and 2.8\% for 5 features). 

In summary, tremors assessment plays an important role in PD diagnoses, but the composition of fluctuations are too complex to conduct direct measurement on body fluctuations. Instead, acceleration features have been studied in previous projects and proved of great value as tremor descriptors in the measurement. 

\section{Conventional Diagnose}
Currently, the most common way for PD diagnoses is periodically collecting reports from patients and manually assessing the materials. Questionnaires from the patients play crucial roles in assessment and some successful scaling methods have been applied in PD progress tracking and pharmacodynamics proving. The most wildly used scales include the Hoen and Yahr\cite{hoehn1998parkinsonism}, the Unique Parkinson's Disease Rating Scale (UPDRS)\cite{movement2003unified} and Northwestern University Disability Scale (NUDS)\cite{goetz2008movement}. The basic principle of these methods is to collect relative information, which is determined by professionals, from patients and scaling critical factors according to patients' description or professionals' observation. As an instance, UPDRS and its revision, Movement Disorder Society-sponsored Unified Parkinson's Disease Rating Scale (MDS-UPDRS), are so far the most widely used assessment criteria in PD diagnoses\cite{goetz2008movement}. In MDS-UPDRS, the marking table consists of four components: Part I spans "non-motor experiences of daily living", Part II concerns the "motor experiences of daily living", Part III and Part IV focus on the motor examination and some complications. For each factor, the severity is marked in 5 levels (0-4). Five types of tremor-relative factors mentioned in Part II and Part IV. In Part II, an overall scaling for all tremors is established while in Part III, factors of tremors are divided into four parts, namely posture tremors, kinetic tremors, rest tremors amplitudes and rest tremor constancy. All items in the table must be answered by patients and the overall marks are believed to roughly present the progress of his/her PD progress.  

Certain advantages of these scales have been identified in these methods. First, little special training is required in this way and in most cases, the information from patients are directly referred to the interesting points of clinicians. Also, it rarely demands expensive instrument, the questionnaires is relatively economy and accessible. However, these methods are still costly, especially on human resources, in a large scale application and suffer from subjectivities and information loss. Medical professionals are always a valuable human resource in society, and frequent diagnoses could be time-consuming for both patients and medical staff. Meanwhile, subjectivity and cognition diversities are unavoidable when patients try to measure and describe the symptoms. As a result, the lack of consistency depresses the reliability of human-based diagnoses. Besides, due to the limitation of memory, it would be difficult for patients to give a completely integral description of symptoms because more or less, daily details could be ignored or forgotten without any conscious. Last but not least, most humans are not capable of giving a precise measurement of physical phenomena, such as frequency and amplitude of tremors, time and duration of action freezing. 

Overall, although conventional diagnoses of PD provides detailed and scientific standards, there are still defects in practice and extensible space in this field.

\section{Diagnose with Advanced Technologies}

To overcome defects aroused by manual works, certain technologies been developed during last decades. An art-of-the-state is the combination of portable devices and machine learning technologies. 

\subsection{Automatic Monitoring}
In contrast with human-based diagnoses, new methods involves embedded sensors, resulting in a superior qualities in integrity, objectivity, economy and accessibility. Portable devices, especially wearable or attachable equipment, are capable of monitoring and recording the physical phenomenon for a long period. 

Recently, with the development of wearable devices, some experiments successfully  achieve nonintrusive and longitudinal monitoring of PD symptoms in portable devices. In a project named SPARK\cite{sharma2014spark}, a multi-layer system combining smartphones and smartwatches were implemented to support remote monitoring of PD symptoms. The pilot work of SPARK suggested promising procedures in remote monitoring. A part of physical variables are collected by smartwatches and then periodically transported to the smartphone via wireless communication. Then the data were upload to the computing cloud, where the analysis algorithms are taken, and personalized interventions would be performed in smartphones in the end. Similar experiments are also carried out by Robert LeMoyne et al. \cite{lemoyne2010implementation}. Differently, an iPhone was mounted in the back of a glove and collects acceleration data by the embedded accelerometer sensor in this experiment. What is more, a research of Giuffrida et al. \cite{giuffrida2009clinically} complete tremor detection with a sensor-embedded ring. Nevertheless, although all 40 participants in their experiment agreed that the wearable devices (a ring and a wrist strap) were comfortable and nonintrusive, only 55\% of them would wear the devices in public. It was suggested that monitoring tremors in a natural way are effective but more natural monitors are expected. 

Instead of extra equipment, smartphones are widely used in daily life. By the end of 2015, the number of smartphones reached 1860 million\cite{website:statista}. In some ways, smartphones have become an essential good for the public. In comparison with special devices, smartphones arouse little impacts on patients daily lives. They are also more accessible and economy. In our pilot study, 17 types of common smartphones, priced from �110 (Blue Studio Energy) to �509 (Samsung Galaxy Note 4), and 5 data collection Android applications were investigated in terms of performance, accessibility and prices\cite{vegaparkinson}. The results indicates that with a proper charging plane, their batteries of smartphones are capable of supporting a whole-day-monitor (except sleeping time). As for data collection platform, four of five applications could collect more than 15 raw features without extra equipment and all of five are free to use. Therefore, portable devices obtain potentials in long-term and continuous monitoring. Moreover, many sensors are designed for objective and accurate detection. Although specifications could be diverse in some extent, they are still more reliable and consistent than human observation and memory. 

According to the former research, it has been ascertained that using acceleration sensors as tremor monitors is promising in PD diagnoses. Nevertheless, this equipment could be costly and the extra devices might cause discomforts and inconvenience in daily lives. And in our pilot work, smartphones were proved to be an ideal monitor without little cost or inconvenience. Hence, smartphones seem like ideal substitution of scientific instruments. However, the sensors of smartphones can not be as accurate as scientific specialties, which results in low data quality. But data cleaning techniques could be set to improve the data quality.

\subsection{Data Technology on Assessment}
Machine learning is a hot point in recent years, which grants systems human-like learning ability. More technically, machine learning techniques are special algorithms designed for capturing information from data. Based on the target, machine learning algorithms could be classified into three classes: classification, regression and clustering. The nature of our project is to reveal the relationship between tremor data and PD progress, which could be seen as a typical regression problem. However, in another way, it could also be seen as a classification problem, i.e. distinguishing PD tremors of different progresses. In this area, some machine-learning-based systems have been proposed.

In the experiment taken by Patel et al.\cite{patel2009monitoring}, an Support Vector Machine (SVM) with a third-order polynomial kernel was developed to recognizing PD tremors. With data collected in a full-controlled laboratory experiment, this system reached a 97\% accuracy. Similarly, Song et al. \cite{pan2012parkinson} reported an experiment aiming at comparing neural networks and SVMs in PD tremor recognition and the result indicated that SVMs produced an better overall classification rate than Multiple Layer Perceptron and Radial Basis Function Network. Another experiment taken by David et al. \cite{gil2009diagnosing} yet combined SVM and Artificial Neural Networks in a PD diagnose system and resulted in higher accuracy around 90\%. The best results came from the experiment held by Daphne et al.\cite{zwartjes2010ambulatory}, which applied a decision tree in PD motor symptom classification and resulted in an overall 99\% accuracy. 

Generally speaking, limited experiments could be found about applying machine learning algorithms in tremor analysis but several inspiring outcomes have been published. These facts suggested the potential of using machine learning techniques in PD tremor measurement. However, all these projects faces limitation in two aspects. First, all these experiments only applied supervised machine learning, which requires heavy labeling work. However, the labeled data in our project is very limited, (17 days). which is barely enough to drive supervised models. Hence, semi-supervised models would be explored. Secondly, these studies mainly focuses on classification of PD patients and healthy subjects instead of predicting and scaling the PD progress. In technical words, most past experiments solved a classification or discriminative problem, not a regression problem. These two methods share a similar distinction that both of them are literally studying certain relationship between inputs and outputs during training. But the main difference here is that classification systems only produced closed outcomes, i.e. predictions must be seen at least once in system training, but a regression system is open and it should be able to produced unseen labels. From another point of view, it is known that classification algorithms usually provide discrete classes while regression algorithms could acquire continuous scores.

\section{Summary}
To sum up, many studies have been taken to explore the characteristics of Parkinson's Disease and the tremor features have been fully explained in relative literature. Based on that, many traditional methods are proposed to manually evaluate the PD progress.  Some advanced systems successfully introduced portable devices and machine learning technique so as to improve the objectivity and integrity of manual ways and deliver automatic classification. Although some of them acquired considerable accuracy indeed, they are not fully functional and the implementation could be expensive and time-consuming.   

%4000---------------------------------------------------------------- Chapter 3 Research Methodology ----------------------------------------------------------------8000%
\chapter{Research Methodology}
\section{Android Sensors and SQLite Database}
\section{Matlab}
%4000---------------------------------------------------------------- Chapter 4 Experiment Design ----------------------------------------------------------------8000%
\chapter{Experiment Design}
\section{Experiment Data Description}
\section{Data Preprocess}
\section{Regression and Classification Models}
\subsection{Logistic Regression}
\subsection{K-NN and K-d Tree}
\subsection{Gaussian Models}
\section{SSL Frameworks}
Semi-supervised machine learning is a sort of algorithms that capable of building classification models with partial labels. Traditional machine learning algorithms demand full-pair data and labels as training materials. However, labels of data are expensive, as it always required massive manual works. On contrast, unlabeled data could be collected purely with sensors and machines, so it is relatively easy to obtain. In data clustering domain, many unsupervised machine learnings have been proposed but not all of them could be used in the classification problem. Semi-supervised machine learning is designed to solve classification problem with partially labeled data. 

Currently, many semi-supervised machine learning algorithms have been used in the pattern recognition domain. The most often-used methods include self-training models, generative models, semi-supervised support vector machines (S3VMs), co-training models and graph-based models. Each of these model types is based on different assumptions in classification, the major point of applying semi-supervise machine learning is looking for a proper assumption that fits the data truth well. 

\subsection{Generative Models}
As the oldest semi-supervised machine learning algorithm\cite{shahshahani1994effect}, the generative model is fairly intuitive and lots of advanced variants have been proposed. The whole algorithm is motivated by building a model to predict $p(x,y) = p(y)p(x|y)$, where $p(x|y)$ presents the possibility of example $x$ belonging to label groups $y$. In practice, we assume this distribution is composed of certain known components, such as Gaussian Mixture Models (GMM)\cite{shahshahani1994effect}, Mixture of Experts\cite{miller1997mixture} and Naive Bayes\cite{miller1997mixture}. With unlabeled data, we can approach these component models. In practice, this algorithm could be embedded into an expectation-maximization (EM) framework, in which the parameters of component models could be iteratively rectified by conducting the maximum likelihood estimation and finally result in an approximation of the ground truth. This method was applied in text classification in 2000 and proved of greater performance than traditional ways\cite{nigam2000text}. Advantages of generative models include: easy to implement with existing GMM functions and better performance with very few labels. Meanwhile, generative model obtains considerable flexibility and adaptability because affluent distribution models could be used as a component model. On the other hand, the usage of unlabeled data will lower its generalization ability when the assumed distribution is improper\cite{cozman2002unlabeled}. 

\subsection{Self-training Models}
In comparison with generative models, self-training models are more intuitive and similar to supervised learning models. In essential, self-training algorithms are an extension of traditional supervised learning methods. First, the training data are divided into two parts, labeled data and unlabeled data. Then certain supervised classifiers are picked up as cores of a self-training model. In each training iteration, the supervised machine learning algorithms will be trained with labeled data and prediction of unlabeled data will be given. Next, the most probable predictions of unlabeled data are accepted as labels and corresponding examples are added to labeled data set. When all unlabeled examples finally obtain the 'possible' label, the iteration will be stopped and the classification is accomplished. For better performance, there is some limitation for the supervised classifiers. First, it must provide the measurable descriptor (possibility) of predictions to conduct the label selection. Moreover, several supervised algorithms require parameter optimization, which would be computationally expensive if the iteration amount is huge, but without optimization, some supervised classifiers might work inefficiently and misclassification will occur frequently. As for merits, similar to generative models, self-training models benefit from the multiformity of model types. Besides, for applications which already have supervised learning versions, the self-training model version could be built easily. 

Many successful trials of self-training models could be found in previous experiments. For example, Riloff and Wiebe reported a highly precise bootstrapping process for sentence subjectivity classifying in 2003\cite{riloff2003learning}. Later in 2005, C Rosenberg et al. proposed a semi-supervised object detection system based on their previous supervised systems\cite{rosenberg2005semi}. An extended application from D McClosky et al. revealed the effects of applying bootstrapping systems as a parser-reranker\cite{mcclosky2006effective}.

\subsection{S3VM Models}
Semi-supervised Support Vector Machine, also known as Transductive Support Vector Machine Learning (TSVM), was firstly brought up by Joachims et al. in 1999\cite{joachims1999transductive}. As well as standard SVMs, S3VMs make an assumption that boundary of clusters is supposed to locate in sparse regions. Hence, SVMs attempt different boundaries and the one with maximum margin in the Reproducing Kernel Hilbert Space is picked as the predicted boundary. Comparing to traditional discriminant classifiers, SVMs consider both $p(y|x)$ and $p(x)$, i.e. the likelihood of instance x having label y and the likelihood of instance x in the model. According to Williams and Seeger, $p(x)$ and $p(y|x)$ must contain shared parameters in the learning models of semi-supervised learning\cite{fine2001efficient}. Hence, SVMs are capable of avoiding traps caused by ignoring the connection of models and instances in comparison with traditional discriminative methods\cite{zhu2005semi}. In S3VMs, predicted labels are estimated when the margin of a boundary is being determined. More specifically, S3VM trains the model with labeled data and give preliminary predictions for unlabeled data, then iteratively switch the labels of unlabeled data to obtain the maximum margin and update the model. As a result, S3VMs find the discriminative boundary that obtains the maximum margin from both labeled data and unlabeled data. However, the loss function of S3VM could be non-convex, resulting in local minimum point in training. This makes the model vulnerable in generation ability. To overcome that, several variants have been proposed to remit the impacts of non-convex loss function in model optimization procedures, such as Deterministic Annealing\cite{sindhwani2006deterministic} and ConCave-Convex Procedure (CCCP)\cite{collobert2006large}.

In practice, J Weston et al. described an experiment based on the S3VM framework proposed by Vapnik's S3VM\cite{chapelletransductive} and the results showed obvious accuracy improvements in comparison to supervised SVM\cite{weston2006inference}. Experiments took by K Bennett and A Demiriz\cite{bennett1999semi} also supported the improvements of S3VM regarding generalization performance.

\subsection{Graph-based Models}
Graph-based models are posed to organize labeled and unlabeled data in a geometry structure, where the edges between nodes refer to the similarity of corresponding examples. The nature of classification graph-based models is label propagation from labeled data to the most specious unlabeled data in geometry views. In essential, graph-based models could be viewed as a compromise between constructing graphs closely reflecting the node labels, i.e. the cluster assumption, and keeping this graph geometric smooth, i.e. the manifold assumption.In general, there are two types of Graph-based semi-supervised classifiers. The most intuitive and natural methods explicitly propagate labels from labeled nodes to unlabeled nodes when certain conditions are satisfied. For example, Zhou X et al. presented an algorithm based on Gaussian Random Field and Harmonic Function\cite{zhu2003semi}. Another method was reported by Zhu D et al. in 2003 which measures the consistency in local and global as the propagation cndition\cite{zhou2004learning}. The implicit methods are implemented by applying regularization in classification to produce similar outcomes for geometrically close examples. The most common functions used in regularization include Mincut Regularizer\cite{blum2001learning}, Tikhonov Regularizer\cite{belkin2004semi} and Manifold Regularizer\cite{belkin2006manifold}. Apart from regularizer functions and loss functions, geometry structure is another critical component that significantly affects the model performance. In past applications, strong understanding of data structure was demanded in model construction. For instance, Balcan et al. introduced a video surveillance system using time series, colors and face features as edges, which is understood by researchers before the experiment. Another limitation of graph-based models, which is also shared in most geometry-based methods, is the computational problem. The computation of edges results in the limitation of nodes amount. According to W Liu, most graph-based models in semi-supervised learning have a time complexity of $O(n^3)$, much higher than that of S3VMs $(O(n^2))$\cite{liu2010large}. So graph-based models would not be a wise choice for applications involving massive data. 

\subsection{Co-training Models}
The co-training model was firstly proposed by Blum and Mitchell in 1998 and some boosted versions were proposed in last decades. Co-training models define two (or more in later versions) weak classifiers in the training stage. In standard co-training algorithm, it makes the assumption that $i)$the training data must obtain at least 2 sufficient and redundant views, i.e. feature sets capable of training a strong classifier, and $ii)$ those views must be conditionally independent, i.e. features are irrelative to each other. Then two classifiers are set to be trained by labeled data with different views (feature sets). In each iteration, the most confident results will be added to the labeled data set and re-train both classifiers in next iteration. These procedures repeat until all data is labeled. Essentially, the whole model makes a compromise of 2 views during the training because each classifier is trained by some instances labeled by another in the previous iteration. Hence, the nature of co-training is mutual teaching between different classifiers. In fact, co-training could be viewed as a combination of self-training and multi-view learning. In other words, if the training views are completely overlapping (the same feature sets), the system will degenerate to a self-training model. The importance of disagreement in weak classifiers have been proved and described by Wang et al\cite{nenkova2011automatic}. However, conditionally independent views would not always exist in practice and it is also difficult to determine the feature set. Herein, some boosted versions of co-training are introduced to relax this constraint. Zhou et al. described a tri-training algorithm using three weak classifiers in training stage\cite{zhou2005tri}, which uses bootstrap, instead of nature independence, to guarantee the disparities of classifiers and a naive voting process in labeled data extension. Li et al. improved the robust of tri-training by replacing the bootstrap algorithm with Random Forest implemented by bagging algorithm\cite{li2007improve}. As a young style, advantages and disadvantages of co-training are not yet explicitly explored. But some experiments have been reported to support them. For example, Xu et al. applied co-training in protein subcellular localization and acquired 10\% higher accuracy over the supervised system\cite{xu2009semi} and Li et al. also successfully implemented a semi-supervised document retrieval system based on co-training learning\cite{li2009semi}.



%8000---------------------------------------------------------------- Chapter 4 Experiment  Implementation ----------------------------------------------------------------11000%
\chapter{Experiment Implementation}
\section{Data Preprocess}
\subsection{Data Clean}
\subsection{Rule Based Filter}
\subsection{Feature Extraction}
Also, the window length of time fragments dramatically affected the accuracy in these experiments and 4s (Patel) and 3s (Daphne G. M. Zwartjes) time window were thought to be suitable for tremor analysis
\section{Model Implementation}
\subsection{Self-training Models}
\subsection{Gaussian Mixture Models}
\section{System Evaluation}
\subsection{Model Evaluation}
\subsection{Feature Evaluation}
%11000---------------------------------------------------------------- Chapter 5 Results and Discussion ----------------------------------------------------------------14000%
\chapter{Experiment Results}
\section{Effects of Data preprocess}
\section{Model Performance}
\section{Feature Assessments}
%14000---------------------------------------------------------------- Chapter 6 Conclusion ----------------------------------------------------------------15000%
\chapter{Conclusion}

%15000---------------------------------------------------------------- Chapter 7 Limitation and Future Work ----------------------------------------------------------------16000%
\chapter{Limitation and Future Work}


\bibliography{PGTDissertation}
\end{document}
